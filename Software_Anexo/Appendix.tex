% Copiar y pegar el contenido en la seccion appendix del template general
\section{Software Description}
This appendix describes the software system developed using the Robot Operating System (ROS) to control the robotic platform designed for the project. The software architecture includes multiple ROS nodes that handle perception, actuation, control, and communication.

\vspace{1em}
\noindent
\textbf{Example:} The software was developed using ROS Noetic on Ubuntu 20.04. It supports a differential-drive robot with autonomous navigation, motor control, sensor integration, and simulation support.

\section*{System Architecture}
The following diagram shows the general architecture of the ROS system, including key nodes, topics, and data flows.

\vspace{1em}
\noindent
\textbf{Node Graph:} \\
\includegraphics[width=\textwidth]{path/to/ros_architecture_diagram.png}

\section*{Node Descriptions}

\begin{table}[H]
\centering
\begin{tabular}{|p{4cm}|p{10cm}|}
\hline
\textbf{Node Name} & \textbf{Function Description} \\
\hline
\texttt{/motor\_controller} & Subscribes to \texttt{/cmd\_vel} and controls wheel motors using PWM signals. \\
\hline
\texttt{/navigation\_node} & Generates velocity commands based on sensor data and goal positions. \\
\hline
\texttt{/sensor\_reader} & Reads LIDAR or ultrasonic data and publishes to \texttt{/scan}. \\
\hline
\end{tabular}
\caption{Key ROS nodes and their functions}
\end{table}

\section*{Topics, Services, and Actions}

\begin{itemize}
    \item \textbf{Topics:} \texttt{/cmd\_vel}, \texttt{/scan}, \texttt{/odom}, \texttt{/goal}.
    \item \textbf{Services:} \texttt{/reset\_odom} (resets odometry).
    \item \textbf{Actions:} \texttt{/move\_base} for goal-directed navigation.
\end{itemize}

\section*{Launch Files}
Describe each launch file used to initialize the system, including configurable parameters.

\vspace{1em}
\noindent
\textbf{Example:} \texttt{main.launch} initializes motor control, sensors, and the navigation stack. Parameters include max speed, serial port, and robot name.

\section*{Hardware Integration}
Explain how the ROS system connects to physical components such as sensors and actuators. Include communication protocols (e.g., I2C, UART, CAN), middleware (e.g., \texttt{rosserial}), or firmware used.

\section*{Simulation Support}
If simulation was used, describe the simulation environment and tools:
\begin{itemize}
    \item Simulator: Gazebo, Webots, etc.
    \item Robot model: URDF/Xacro file locations and descriptions.
    \item Plugins and controllers used.
\end{itemize}

\section*{Testing and Validation}
Summarize how the software was tested:
\begin{itemize}
    \item Unit tests for individual nodes or scripts.
    \item Integration tests with the full robot.
    \item Real-world functional tests and observations.
\end{itemize}

\section*{Installation and Usage Instructions}

\begin{itemize}
    \item \textbf{Dependencies:} ROS Noetic, Python 3, packages: \texttt{rospy}, \texttt{geometry\_msgs}, \texttt{tf2}, etc.
    \item \textbf{Build:} \texttt{catkin\_make}
    \item \textbf{Run:} \texttt{roslaunch my\_robot\_package main.launch}
\end{itemize}

\section*{Code Repository}
If the code is available online, provide a link:

\noindent
\texttt{https://github.com/username/robot-project}